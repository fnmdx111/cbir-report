\documentclass[12pt, titlepage]{ctexrep}

\usepackage{amsmath}
\usepackage{amsfonts}

\usepackage[hidelinks]{hyperref}

\usepackage{fontspec}
\usepackage{minted}

\setmainfont{华文宋体}
\setmonofont{Courier New}
\author{武汉大学对不队}
\title{生物特征图像安全检索系统}

\pagestyle{headings}
\maketitle
\begin{document}

\tableofcontents

\chapter{摘要}
\label{chap:abstract}

\chapter{作品介绍}
\label{chap:intro}

\section{背景分析}
\label{sec:bkg-analysis}

\section{特色描述}
\label{sec:spec-description}

传统的内容检索系统不能支持对于密文的检索,而传统的加密存储系统不能支持内容检索。
本作品从JPEG文件的压缩算法入手,
实现了一个基于Logistics混沌系统的置乱加解密算法和基于离散余弦变换的内
容检索系统,
并基于这两个算法,实现了一个客户端/服务端构架的系统。
其中客户端/服务端都由两部分(内核和前端)组成。

\subsection{系统方案特色}
\label{sec:sys-design-spec}
\begin{itemize}
\item 加密内容写回至JPEG文件,使加密内容可见,但不可理解。
\item 加密算法使用的是像素置换和,加解密  速度较快。
\item 由于Logistics映射的优良性质\cite{li2011},加密算法的密钥空间至少高于128bit。
\item 向量距离作为内容检索标准,速度较快。
\end{itemize}

\subsection{系统实现特色}
\label{sec:sys-impl-spec}
\begin{itemize}
\item 整个系统构建于经过广大用户验证过的开源库或开源程序上,可以保证软
  件基础设施的安全性。
\item 系统构架为客户端/服务端,其中交互使用经过广泛使用的HTTP协议,使
  一个服务器可以支持多个客户端同时访问,同时不增加服务端的程序复杂性。
\item 服务端使用简洁的Flask框架和支持高并发的tornado库,高效可靠。
\item 客户端使用了跨平台的Qt框架作为前端,内核算法使用了实现了多种快速
  算法的OpenCV库和numpy库。
\item 客户端服务端的内核与前端分开,耦合度低,用户可以方便的进行二次开
  发。
\end{itemize}

\section{相关工作}
\label{sec:related-work}

为了实现系统设计目标,我们研究了离散余弦变换和Logistic映射的性质及其应用。

\subsection{离散余弦变换}

离散余弦变换(以下简称DCT)是一种将时空域上的函数转换到频域的、具有良好能
量聚集性质的、存在快速算法的数学变换。其公式如下:
\begin{displaymath}
X_k = \frac1 2 \left[x_0 + \left(-1\right)^k x_{N - 1}\right] + \sum_{n = 1}^{N - 2} x_n \cos \left(
\frac{\pi} {N - 1} n k \right) & \qquad k = 0, \dotsc, N - 1
\end{displaymath}

JPEG图片格式利用了人眼对低频信号敏感、对高频信号不敏感的特点对高频信号
进行压缩。DCT使用最广泛的一种变形——二维DCT可以将良好的将高频与低频信
号分离。在JPEG编码中,整个图像数据(灰度值)按$8 \times 8$分块,每块经
过离散余弦变换、量化和熵编码之后按照JPEG标准编码到文件中。

二维DCT(DCT-II)公式如下:
\begin{displaymath}
X_k = \sum_{n = 0}^{N - 1} x_n \cos \left[\frac{\pi} N \left(n +
\frac1 2\right) k\right] & \qquad k = 0, \dotsc, N - 1
\end{displaymath}

在DCT-II中,输入矩阵$I$的能量被充分聚集到了左上角,其中输出矩阵$M$的
$M_{0, 0}$是最能代表输入矩阵的值,这个值被称为DC系数(直流系数),即输入
波形的均值。矩阵中其他的值被称为AC系数(交流系数)。
离DC系数越远的AC系数所代表的频率就越
高,人眼也就对它越不敏感。



\subsection{Logistic映射}

Logistic映射(或称单峰映射)是一种二次的多项式映射,是实际系统中存在的
最简单的非线性差分方程之一\cite{yang2011}。这个映射因生物学家Robert
May在1976年发表的一篇论文而著名。其公式如下:
\begin{displaymath}
  x_{n + 1} = rx_n(1 - x_n) & \qquad 0 < r \leq 4, r \in \mathbb{R}
\end{displaymath}

Logistic映射所构建的密钥序列具有良好的随机性和初值敏感性,当$r >
3.569945672\dotsb$时,系统进入混沌状态,产生的序列非周期不收敛,随着初
始值的不同有着非常大的差异性\cite{yang2011}。

混沌现象是在非线性动力系统中出现的、具有对初始条件的
敏感依赖性、类噪声、非周期性、确定性的、类随机的过程,这种过程既非周期
又不收敛,其状态完全可以重现\cite{lu2007}。故本系统选用该映射来生成混
沌序列,随后用该混沌序列来生成初始密钥敏感的伪随机序列,
最后利用多个伪随机序列构成置乱矩阵来加密输入图片。

\section{市场分析}
\label{sec:market-analysis}



\chapter{系统设计}
\label{chap:sys-design}

\section{系统说明}
\label{sec:sys-description}

\section{设计目标}
\label{sec:design-goal}

\section{系统架构}
\label{sec:sys-arch}

系统由客户端和服务端组成。

\subsection{客户端}
客户端由前端模块(\texttt{ui.py})和核心模块(\texttt{libs}包)构成,
而核心模块又由\texttt{ClientCore}类(\texttt{libs/core.py})和
\texttt{LogisticPermutation}类(\texttt{libs/logistic.py})组成。其中
\begin{itemize}
\item \texttt{ClientCore}类负责图像的加解密与服务器的交互(例如上传图
  片至数据库或请求检索图片等);
\item \texttt{LogisticPermutation}类负责根据给定的初始密钥求出加解密置
  换表。
\end{itemize}

\subsection{服务端}

\section{功能原理及实现}
\label{sec:func-impl}

为了达到标准化图片库,便于实现功能的目的,本系统只接受宽$640$像素,高
$480$像素,即$4:3$的灰度JPEG图片。

\subsection{检索匹配方面}
\label{sec:retrieval-impl}
本系统利用二维DCT抽取出一张图片的特征向量。具体方法如下:
\begin{enumerate}
\item 将输入JPEG图片按$8 \times 8$分块,对每块做DCT;
\item 取每块的DC系数,添加到数组$V$中;
\item 将$V$按降序排序,得到$W$,并取$W$的前$n$个元素作为该输入图片的特征向量。
\end{enumerate}

其中$n$可以根据实际需要来取。由于$W$中下标越大的元素,其值越小,在向量
距离计算中起到的作用就越小,因此本系统实现中对检索效果和效率进行衡量,
选择了$128$作为该参数的值。

计算距离方面使用的是numpy库中的linalg.norm函数。

\subsection{加解密方面}
\label{sec:enc-dec-impl}

本系统使用的加解密算法的初始密钥由三组$x_0$,$r$,$s$组成,
其中$x_0 \in (-1, 1)$,$r \in [3.57, 4]$,$s > 0, s \in \mathbb{Z}$。

加密算法首先使用伪随机序列生成算法生成若干置乱表,然后根据置乱表对图片
进行块内和块间的置乱达到使图片不可理解、但保留DC系数的统计特性的目的。
解密算法与加密算法类似,只是使用的置乱表不一样,即加解密算法是对合的。

由Logistic映射生成具有$n$个元素的伪随机序列的具体做法\cite{lu2007}是:
\begin{enumerate}
  \item 使用初始密钥$x_0$和$r$,迭代Logistic公式,生成具有$n + s$个元
      素的混沌序列,并舍去前$s$个元素,得到$L = \{x_1, x_2, x_3, \dotsc, x_n\}$;
  \item 对$L$进行升序排序,得到$M = \{x_1^\prime, x_2^\prime, x_3^\prime,
      \dotsc, x_n^\prime\}$;
  \item 定位$x_i$在$x_i^\prime$中的位置序数,得到序数序列,记为$P =
      \{p_1, p_2, p_3, \dotsc, p_n\}$
\end{enumerate}
$P$就是所求的关于初始密钥$x_0$、$r$和$s$的伪随机序列。

由置乱表生成逆置乱表的具体做法是:
\begin{enumerate}
  \item 扩展输入序列为$P^\prime = \{(0, x_1), (1, x_2), (2, x_3),
      \dotsc, (n - 1, x_n)\} & \quad x_i \in P$;
  \item 建立一个新表$I$,使得$\forall (i, x_{i + 1}) \in P^\prime$,
      $I_{x_{i + 1}} = i$
\end{enumerate}
$I$就是所求的关于$P$的逆置乱表。

本系统为了保证每个DCT块的均值不变,利用不同的初始密钥生成了3个伪随机序列,分别是:
\begin{itemize}
  \item 利用第一组初始密钥生成了具有$w$个元素的$T_{column}$,作为块间的列置乱表;
  \item 利用第二组初始密钥生成了具有$h$个元素的$T_{row}$,作为块间的行置乱表;
  \item 利用第三组初始密钥生成了具有64个元素的$T_{block}$,作为DCT块内的置乱表;
\end{itemize}
其中$w$等于图片宽度除以$8$,$h$等于图片宽度除以$8$。

\chapter{性能测试}
\label{chap:benchmark}

\section{测试指标}
\label{sec:benchmark-index}

\section{测试方案}
\label{sec:benchmark-scheme}

\section{测试环境}
\label{sec:benchmark-env}

\section{测试数据}
\label{sec:benchmark-data}

\section{测试结果及分析}
\label{sec:result-and-analysis}

\chapter{创新性}
\label{chap:creativity}

\chapter{总结}
\label{chap:summary}


\begin{thebibliography}{9}
\bibitem{li2011} 李进, 徐红.
 \emph{基于MD5算法和Logistic映射的图像加密方法研究}[J]. 信息网络安全,
 2011, (8): 25 - 26, 47.
\bibitem{yang2011} 杨恒欢, 冯涛, 荆锐. \emph{一种黑洞数和Logistic混沌序列的图像加密
  应用}[J]. 上海第二工业大学学报, 2011, 28(4): 287.
\bibitem{lu2007} 陆大兴,廖晓峰,韩洁等. \emph{基于Logistic映射与排序变
    换的图像加密算法}
[J].计算机技术与发展,2007,17(12):27-30.
\end{thebibliography}

\end{document}
