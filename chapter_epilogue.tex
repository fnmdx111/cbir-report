
\chapter{总结}
\label{chap:summary}

\section{创新性}
\label{sec:creative-points}

多媒体生物特征数据需要安全存储,同时也要求可以支持内容检索功能。传统加
密处理后的数据一般无法有效进行内容检索,更无法找寻同类或相似的图像或其
它生物特征数据。

基于加密数据库的内容检索,其目标是提供关于加密文档的、高效准确的搜索能
力,并且在搜索时应不需要先解密文档。本系统采用不改变图像特征统计特性的
方法、对特定编码格式的图像文件(JPEG格式)进行特殊的加密处理,算法具有
快速性、简单性、安全性、鲁棒性等性质。

本系统可以实现具有一般权限的操作员在无法获知明文图像的条件下,对多媒体
生物特征数据进行内容检索和管理,能够搜索到相同内容或近似内容的对象数据。
该安全检索系统可以容忍多数的不影响内容的多媒体处理操作,并且其检索性能
不受加密处理影响。

\section{工作综述}
\label{sec:work-overview}

本系统涉及到数学、密码学、图像编码与网络通信等多个学科领域,涵盖了离散
余弦变换、Logistic混沌系统等知识,运用了numpy、opencv等关键技术,针对内
容检索系统的明文内容可能被非法使用等安全问题,设计并实现了一个基于内容
的生物特征图像安全检索系统。

项目团队于2013年2月成立,至今约4个月的时间。在这4个月中,项目成员在大量
课业压力下和指导老师的指导下,进行相关背景知识的调研和学习、系统的设计、
开发与测试,文档的编写与修改。

由于项目成员均是第一次参与到具有较多专业知识系统的开发中、对专业知识的
储备较少,而且课业繁重,项目一度处于停滞状态。但在指导老师的帮助下,项
目成员自发地投身于相关论文的阅读,相关函数库的自学,和程序代码的编制中。
在牺牲了大量的休息娱乐时间之后,较好的完成了设计目标,实现了一个具有良
好图形用户界面的、稳定且快速的系统。

本次项目开发中,项目组成员自学了多种技术手段,并以大量代码的形式,将它
们运用到实践中,锻炼了动手能力,体会到了理论与实践之间不小的差距。在文
档的撰写中也练习了平时很少有机会运用的文档写作技巧及相关软件。

\section{后续工作}
\label{sec:next-step}

由于能力与时间的限制,本系统功能未能做到完善,需要做进一步的调查与研究。
不足主要在以下几方面:
\begin{itemize}
\item 寻求可在保留检索用统计特性的同时,置乱无关统计特性的方法并实现之;
\item 寻找更好的客户端密钥管理方案并实现之;
\item 服务端管理功能不足,增加必要的图像元信息以便对密图集做更细致的管理。
\end{itemize}
