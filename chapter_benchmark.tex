
\chapter{性能测试}
\label{chap:benchmark}

为了测试本系统的可用性,需要测试检索算法的鲁棒性和效率。

\section{测试指标}
\label{sec:benchmark-index}
鲁棒性:本系统利用DCT抽取出图像的特征向量,采用向量间的距离作为图片的
相似程度的度量,距离越大,图片相似程度越低,同理,距离越小(大于0),
图片的相似程度就越高。对鲁棒性的测试应着眼于在不同情况下,加噪图与
原图的距离的变化情况,距离越小,抗噪声性能越好。

效率:本系统接受一个待检索图后,在特征数据库中检索所有的特征向量后返回
前10相似的图片。对效率的测试应着眼于系统检索出结果的时间,以及客户端接
收到全部结果图片的时间,检索时间越小,服务端检索性能越好,接受时间越
小,客户端性能越好。

\section{测试方案}
\label{sec:benchmark-scheme}

\begin{itemize}
\item 为了测试检索算法的鲁棒性,拟选择一些样本图片,并对这些样本做一些典型的
变化:
\begin{itemize}
\item 加入椒盐噪声
\item 改变质量因子$Q$
\end{itemize}
然后统计加噪图与原图的距离。

\item 为了测试检索算法的效率,拟在同一台主机上同时开启本系统的服务端和客户端,
即在忽略网络状况的情况下测试系统的检索速度并统计时间。
\end{itemize}

\section{测试环境}
\label{sec:benchmark-env}

\begin{itemize}
\item 硬件环境:
\begin{itemize}
\item 处理器:Intel Core i5-2410M CPU @ 2.30GHz
\item 内存:4.00GB(3.82GB可用)
\end{itemize}
\item 软件环境:
\begin{itemize}
\item 操作系统:Windows 8 专业版 x64(含Media Center) 
\item Python 2.7.3 x86
\item numpy 1.7.1
\item opencv-python-2.4.5
\end{itemize}
\end{itemize}

\section{测试数据}
\label{sec:benchmark-data}

测试图像集为37张灰度人脸图,大小为$640 \times 480$,单位为像素。
样图如下:


\section{测试结果及分析}
\label{sec:result-and-analysis}

\subsection{椒盐噪声}
\label{sec:speckle-data}

典型的加入椒盐噪声的图如下

对上示原图加入噪声密度在$[0, 0.1]$内、步长为0.01的椒盐噪声后,分别统计
与原图的距离,可作如下的散点图:

不失一般性,选取37张图片做同样处理,统计出散点图,如图:

从图中可以看出来,本算法可以抵抗噪声密度小于$0.1$的椒盐噪声,在此噪声
密度下,加噪图与原图距离普遍小于$500$,当噪声密度大于$0.1$后,加噪图
与原图距离增大到了无法接受的地步。

\subsection{质量因子}
\label{sec:quality-data}

原图与质量因子$Q = 50$的图如下

调整上示原图的质量因子$Q \in [0, 0.9]$、步长为0.1后,分别统计与原图的
距离,可作如下的散点图:

不失一般性,选取37张图片做同样处理,统计出散点图,如图:

从图中可以看出来,本算法有较好的抗压缩性能,即使图片质量被调整为$10\%$,
与原图的距离仍然小于$500$。


