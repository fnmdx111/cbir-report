特色描述 ======

传统的内容检索系统不能支持对于密文的检索,而传统的加密存储系统不能支持内容检索。本作品从JPEG文件的压缩算法入手,实现了一个基于Logistics混沌系统的置乱加解密算法和基于离散余弦变换的内容检索系统,并基于这两个算法,实现了一个客户端/服务端构架的系统。其中客户端/服务端都由两部分------内核和前端------组成。

系统方案特色 ---------

* 加密内容写回至jpeg文件,使加密内容可见,但不可理解。 *
加密算法使用的是O(n\textsuperscript{2})的置换和O(nlogn)的离散余弦变换,速度较快。
* (可能可以实现对dct矩阵的异或加密) *
由于Logistics序列的优良性质,加密算法的密钥空间至少高于128bit。 *
向量距离作为内容检索算法,速度较快。

系统实现特色 ---------

*
整个系统全部使用经过广大用户验证过的开源程序或开源库实现,本系统在GPL下发布,可以保证软件基础设施安全性。
*
系统构架为客户端/服务端构架,CS交互使用http协议,一个服务器可以支持多个客户端同时访问,服务器使用简介的flask框架和高并发的tornado库,高效可靠。
*
可移植性和良好的可维护性:使用了Python编程语言,可以支持多种平台,客户端前端使用了Qt框架的Python绑定,可以在多种平台上运行,算法内核使用了c语言实现的快速高效的opencv和numpy库。
*
客户端服务端的核心与前端分开,使得用户可以自行使用预留的api进行二次开发。

实现原理 ======

离散余弦变换 ---------

离散余弦变换(一下简称DCT)是一种将时空域数据转换到频域的、与离散傅里叶变换一样是具有良好的能量聚集性质的、有快速算法的数学变换。JPEG图片格式利用了人眼对低频信号敏感、对高频信号不敏感的特点对高频信号进行压缩,而二维DCT可以良好的将高频与低频信号分离。在JPEG图片格式中,整个图像数据(灰度值)按8x8分块,每块经过离散余弦变换、量化和熵编码之后按照JPEG标准编码到文件中。

DCT公式如下:

2D-DCT公式如下:

在2D-DCT中,输入矩阵的能量被充分聚集到了左上角,而输出矩阵的(0,
0)则是最能代表输入矩阵的一个值,这个值被称为DC系数(直流系数),参见2D-DCT公式,DC系数实际上是某种形式的平均值。矩阵中其他的值被称为AC系数(交流系数),其中离DC系数越远,其代表的频率就越高,人眼也就对它越不敏感。

本系统利用二维DCT变换抽取出一张图片的特征。具体采用了如下方法

1. 将输入图片按8x8分块,每块做DCT 2. 取每个输出块的DC系数
-\textgreater{} L(DCT) 3.
将L(DCT)排序,并输出前128个元素作为该输入图片的特征向量

元素个数与检索质量(图像距离)间的单调性分析?

Logistics混沌系统是xxx
