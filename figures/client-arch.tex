
\begin{figure}[H]
\centering
\begin{tikzpicture}[
  level distance=130pt,
  font=\sffamily,
  basic/.style={draw,text width=2cm,drop shadow,font=\sffamily,rectangle},
  root/.style={basic,rounded corners=2pt,thin,align=center,fill=green!30},
  level 2/.style={basic,rounded
    corners=6pt,thin,align=center,fill=green!50,sibling distance=5mm,
    text width=6em},
  level 3/.style={basic,thin,align=left,fill=cyan!20,text width=10em},
  level 1/.style={sibling distance=10mm,level distance=100pt, text width=5em},
  edge from parent/.style={->,draw},
  grow'=right,
  >=latex]

\Tree [.\node[root] {客户端};
          [.\node[level 2] {前端};
              [.\node[level 3] (ui-node) {\texttt{ui.py}}; ]
              [.\node[level 3] (ui-comp-node)
              {\texttt{libs/ui\_comp.py}}; ] ]
          [.\node[level 2] {核心};
              [.\node[level 3] (core-node) {\texttt{libs/core.py}}; ]
              [.\node[level 3] (logistic-node) {\texttt{libs/logistic.py}}; ] ] ]
\end{tikzpicture}
\caption{客户端架构图示}
\label{fig:client-arch}
\end{figure}

