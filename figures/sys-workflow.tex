% Data flow diagram
% Author: David Fokkema
\pgfdeclarelayer{background}
\pgfdeclarelayer{foreground}
\pgfsetlayers{background,main,foreground}

\begin{figure}[H]

\centering
\begin{tikzpicture}[
font=\sffamily,
    every matrix/.style={ampersand replacement=\&,column sep=1cm,row sep=1cm},
    source/.style={draw,thick,rounded corners,fill=red!20,inner sep=.3cm},
    process/.style={draw,thick,rounded corners,inner sep=.3cm,fill=blue!20},
    sink/.style={source,fill=green!20},
    to/.style={->,>=stealth',shorten >=1pt,semithick,font=\sffamily\footnotesize},
    every node/.style={align=center}]

    % Position the nodes using a matrix layout
    \matrix{
        \& \& \node[source] (inputimg) {录入图像}; \& \\
        \& \& \node[process] (correction) {矫正}; \& \\
        \node[source] (restoredimg) {还原图};
        \& \node (hiddennodeclient) {};
        \& \node[source] (ainputimg) {待添加图像};
        \& \node[source] (rinputimg) {待检索图像}; \\
          \node[process] (decimg) {解密 (逆置乱)};
        \& \& \node[process] (encimg) {加密 (置乱)};
        \& \node[process] (encimga) {加密 (置乱)}; \\
    \node (hiddenhttpa) {}; \& \node (hiddenhttpb) {};
      \& \node (hiddenhttpc) {}; \& \node (hiddenhttpd) {}; \\
    \node (hiddennodeserver) {};
       \& \& \node[process] (extfeat) {提取特征}; 
       \& \node[process] (extfeata) {提取特征}; \\
    \node[source] (results) {匹配结果};
        \& \node[sink] (imgdb) {图片数据库};
        \& \node[sink] (featdb) {特征数据库};
        \& \node (rightbottom) {}; \\
  };

  \path (hiddennodeclient.south) + (0, -0.5) node (client)
      {\Large \emph{客户端}};
  \path (hiddennodeserver.east) + (1.8, -0.2) node (server) {\Large
     \emph{服务端}};
  \path (hiddenhttpa.east) + (1.8, 0) node (http-level)
      {\Large \emph{HTTP}};

  % Draw the arrows between the nodes and label them.
  \draw[to] (decimg) -- (restoredimg);
  \draw[to] (results) -- (decimg);
  \draw[to] (inputimg) -- (correction);
  \draw[to] (correction) -- (ainputimg);
  \draw[to] (ainputimg) -- (encimg);
  \draw[to] (encimg) -| node[midway, above] {加入} (imgdb);
  \draw[to] (encimg) -- (extfeat);
  \draw[to] (extfeat) -- node[midway, left] {加入} (featdb);
  \draw[to] (rinputimg) -- (encimga);
  \draw[to] (encimga) -- (extfeata);
  \draw[to] (extfeata) |- node[midway, below] {比对} (featdb);
  \draw[to] (featdb) -- (imgdb);
  \draw[to] (imgdb) -- (results);

  \begin{pgfonlayer}{background}
    \path (decimg.west |- rinputimg.north) + (-0.3, 0.3) node
    (a) {};
    \path (encimg.south -| rinputimg.east) + (+0.3, -0.3) node (b)
    {};
    \path[fill=cyan!10, rounded corners, draw=black!50, dashed] (a)
    rectangle (b);
    \path (extfeat.north -| extfeata.east) + (+0.5, +0.3) node (a) {};
    \path (results.west |- imgdb.south) + (-0.7, -0.3) node (b) {};
    \path[fill=yellow!10, rounded corners, draw=black!50, dashed] (a)
    rectangle (b);
    \path (hiddenhttpa.west |- hiddenhttpd.north) + (-1.7, 0.3) node
    (a) {};
    \path (hiddenhttpc.south -| hiddenhttpd.east) + (+1.5, -0.3) node
    (b) {};
    \path[fill=gray!10, rounded corners, draw=black!50, dashed] (a)
    rectangle (b);
  \end{pgfonlayer}

\end{tikzpicture}

\caption{系统流程图示}
\label{fig:sys-workflow}
\end{figure}

