\begin{center}
  \begin{longtable}[H]{| l | p{3.5cm} | p{3cm} | p{4cm} |}
    \hline
    \textbf{方法名} & \textbf{参数} & \textbf{返回} & \textbf{说明} \\
    \hline
    \endfirsthead
    \multicolumn{4}{r}{\textit{接上页}} \\
    \hline
    \textbf{方法名} & \textbf{参数} & \textbf{返回} & \textbf{说明} \\
    \hline
    \endhead
    \hline
    \endfoot
    \endlastfoot
    \texttt{open\_img} & \texttt{path:} 图像路径
  & \texttt{np.ndarray}型的灰度矩阵表示打开的图像 & 用于打开给定路径的JPEG文件 \\
    \hline
    \texttt{save\_img} & \texttt{path:} 保存路径\newline
    \texttt{img:} 灰度矩阵 & 无 & 用于按JPEG格式保存给定的灰度矩阵 \\
    \hline
    \texttt{\_transform} & \texttt{img:} 灰度矩阵\newline
                           \texttt{table:} 块间置乱表\newline
                           \texttt{block\_table:} 块内置乱表\newline
  & \texttt{np.ndarray}型的按照给定的表置乱后的矩阵 & 主置乱函数,\newline
    \texttt{enc\_img}与\texttt{dec\_img}\newline
    都是在这个方法的基础上实现的 \\
    \hline
    \texttt{upload\_img} & \texttt{raw:} 待上传图像的二进制数据
  & 服务器返回的json & 用于上传图像 \\
    \hline
    \texttt{send\_img} & \texttt{raw:} 待检索的二进制数据\newline
                         \texttt{max\_count:} 最大结果数,默认为10
  & 服务器返回的json & 用于检索图像 \\
    \hline
    \texttt{parse\_result} & 无 & 服务器返回图像的二进制数据与该图像与
    原图的距离 & 用于获取服务器准备好的结果,每次调用获取一个 \\
    \hline
    \caption{\texttt{ClientCore}主要方法} \\
  \end{longtable}
  \label{tab:client-core}
\end{center}

