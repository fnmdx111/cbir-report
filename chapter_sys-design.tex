
\chapter{系统设计}
\label{chap:sys-design}

\section{设计目标}
\label{sec:design-goal}

提出一个实用的生物特征图像安全检索系统用到的检索与加解密算法,并由可靠
的部件与模块实现它,在实现典型的基于内容的图像检索系统的基础上,确保系
统内图像的内容安全性,其检索性能不受加密处理影响。

系统的安全性即,一般权限人可以对图像检索系统进行数据上的管理,如取出系
统中的密图、或密图的特征向量并利用密图的特征向量进行图像匹配等,但一般
权限人无法理解图像内容,只有图像所有人利用加密密钥计算出解密序列对图像
解密后才能获得可以理解的图像。

\section{系统说明}
\label{sec:sys-description}

% Data flow diagram
% Author: David Fokkema
\pgfdeclarelayer{background}
\pgfdeclarelayer{foreground}
\pgfsetlayers{background,main,foreground}

\begin{figure}[H]

\caption{系统架构图示}
\label{fig:sys-arch}
\centering
\begin{tikzpicture}[
font=\sffamily,
    every matrix/.style={ampersand replacement=\&,column sep=1cm,row sep=1cm},
    source/.style={draw,thick,rounded corners,fill=red!20,inner sep=.3cm},
    process/.style={draw,thick,rounded corners,inner sep=.3cm,fill=blue!20},
    sink/.style={source,fill=green!20},
    to/.style={->,>=stealth',shorten >=1pt,semithick,font=\sffamily\footnotesize},
    every node/.style={align=center}]

    % Position the nodes using a matrix layout
    \matrix{
        \& \& \node[source] (inputimg) {录入图像}; \& \\
        \& \& \node[process] (correction) {矫正}; \& \\
        \node[source] (restoredimg) {还原图};
        \& \node (hiddennodeclient) {};
        \& \node[source] (ainputimg) {待添加图像};
        \& \node[source] (rinputimg) {待检索图像}; \\
          \node[process] (decimg) {解密 (逆置乱)};
        \& \& \node[process] (encimg) {加密 (置乱)};
        \& \node[process] (encimga) {加密 (置乱)}; \\
    \node (hiddenhttpa) {}; \& \node (hiddenhttpb) {};
      \& \node (hiddenhttpc) {}; \& \node (hiddenhttpd) {}; \\
    \node (hiddennodeserver) {};
       \& \& \node[process] (extfeat) {提取特征}; 
       \& \node[process] (extfeata) {提取特征}; \\
    \node[source] (results) {匹配结果};
        \& \node[sink] (imgdb) {图片数据库};
        \& \node[sink] (featdb) {特征数据库};
        \& \node (rightbottom) {}; \\
  };

  \path (hiddennodeclient.south) + (0, -0.5) node (client)
      {\Large \emph{客户端}};
  \path (hiddennodeserver.east) + (1.8, -0.2) node (server) {\Large
     \emph{服务端}};
  \path (hiddenhttpa.east) + (1.8, 0) node (http-level)
      {\Large \emph{HTTP}};

  % Draw the arrows between the nodes and label them.
  \draw[to] (decimg) -- (restoredimg);
  \draw[to] (results) -- (decimg);
  \draw[to] (inputimg) -- (correction);
  \draw[to] (correction) -- (ainputimg);
  \draw[to] (ainputimg) -- (encimg);
  \draw[to] (encimg) -| node[midway, above] {加入} (imgdb);
  \draw[to] (encimg) -- (extfeat);
  \draw[to] (extfeat) -- node[midway, left] {加入} (featdb);
  \draw[to] (rinputimg) -- (encimga);
  \draw[to] (encimga) -- (extfeata);
  \draw[to] (extfeata) |- node[midway, below] {比对} (featdb);
  \draw[to] (featdb) -- (imgdb);
  \draw[to] (imgdb) -- (results);

  \begin{pgfonlayer}{background}
    \path (decimg.west |- rinputimg.north) + (-0.3, 0.3) node
    (a) {};
    \path (encimg.south -| rinputimg.east) + (+0.3, -0.3) node (b)
    {};
    \path[fill=cyan!10, rounded corners, draw=black!50, dashed] (a)
    rectangle (b);
    \path (extfeat.north -| extfeata.east) + (+0.5, +0.3) node (a) {};
    \path (results.west |- imgdb.south) + (-0.7, -0.3) node (b) {};
    \path[fill=yellow!10, rounded corners, draw=black!50, dashed] (a)
    rectangle (b);
    \path (hiddenhttpa.west |- hiddenhttpd.north) + (-1.7, 0.3) node
    (a) {};
    \path (hiddenhttpc.south -| hiddenhttpd.east) + (+1.5, -0.3) node
    (b) {};
    \path[fill=gray!10, rounded corners, draw=black!50, dashed] (a)
    rectangle (b);
  \end{pgfonlayer}

\end{tikzpicture}


\end{figure}


如图\ref{fig:sys-arch}所示,系统由客户端和服务端构成。

其中客户端所实现的功能有:
\begin{itemize}
\item 加密(置乱图像)
\item 解密(逆置乱图像)
\end{itemize}

服务端所实现的功能有:
\begin{itemize}
\item 提取输入图像的特征
\item 将密图加入图片数据库
\item 将特征加入特征数据库
\item 比对特征数据库中的特征
\end{itemize}

客户端与服务端通过HTTP协议进行交互,必要时可以使用HTTPS协议来增加安全
性而无需修改大量代码。

\section{系统架构}
\label{sec:sys-arch}
\subsection{客户端}


\begin{figure}[H]

\centering
\begin{tikzpicture}[
level distance=130pt,
font=\sffamily,
  basic/.style={draw,text width=2cm,drop shadow,font=\sffamily,rectangle},
  root/.style={basic,rounded corners=2pt,thin,align=center,fill=green!30},
  level 2/.style={basic,rounded
    corners=6pt,thin,align=center,fill=green!50,sibling distance=5mm,
    text width=6em},
  level 3/.style={basic,thin,align=left,fill=pink!60,text width=10em},
  level 4/.style={basic,thin,align=left,fill=cyan!20,text width=12em,
    level distance=100pt},
  level 1/.style={sibling distance=10mm,level distance=100pt, text width=5em},
  edge from parent/.style={->,draw},
  grow'=right,
  >=latex]

\Tree [.\node[root] {客户端};
          [.\node[level 2] {前端};
              [.\node[level 3] (ui-node) {\texttt{ui.py}}; ]
                %  [.\node[level 4] {\texttt{SecureImageRetrieval}}; ] ]
              [.\node[level 3] (ui-comp-node)
              {\texttt{libs/ui\_comp.py}}; ] ]
                  % [.\node[level 4] {\texttt{ResultListItemDelegate}}; ]
                  % [.\node[level 4] {\texttt{ResultListModel}}; ] ] ]
          [.\node[level 2] {核心};
              [.\node[level 3] (core-node) {\texttt{libs/core.py}}; ]
                  % [.\node[level 4] {\texttt{ClientCore}}; ] ]
              [.\node[level 3] (logistic-node) {\texttt{libs/logistic.py}}; ] ] ]

% \begin{scope}[every node/.style={level 3}]
%   \node [below of = ui-node, xshift=15pt] (ui-n-1)
%    % {\mint{python}|class SecureRetrievalUI(QDialog, object)|};
%     {SecureRetrievalUI};
%   \node [below of = ui-comp-node, xshift=15pt] (ui-c-n-1)
%    % {\mint{python}|class ResultListItemDelegate(QStyledItemDelegate,
%    %   object)|};
%     {ResultListItemDelegate};
%   \node [below of = ui-c-n-1] (ui-c-n-2)
%    % {\mint{python}|class ResultListModel(QAbstractListModel,
%    % object)|};
%     {ResultListModel};
%   \node [below of = ui-c-n-2] (ui-c-n-3)
%   %    {\mint{python}|class ImageWidget(QWidget, object)|};
%     {ImageWidget};
%   \node [below of = ui-c-n-3] (ui-c-n-4)
%   %    {\mint{python}|class LoggerHandler(Handler)|};
%     {LoggerHandler};
%   \node [below of = ui-c-n-4] (ui-c-n-5)
%   %    {\mint{python}|class ColoredFormatter(Formatter)|};
%     {ColoredFormatter};
% \end{scope}

% \foreach \value in {1,2}
%   \draw[->] (lvl21.195) |- (lvl31\value.west);
%  \foreach \value in {1,2}
%   \draw[->] (lvl22.195) |- (lvl32\value.west);
\end{tikzpicture}
\caption{客户端架构图示}
\label{fig:client-arch}
\end{figure}


客户端由前端模块(\texttt{ui.py})和核心模块(\texttt{libs}包)构成,
而核心模块又由\texttt{ClientCore}类(\texttt{libs/core.py})和
\texttt{LogisticPermutation}类(\texttt{libs/logistic.py})组成。其中
\begin{itemize}
\item \texttt{ClientCore}类负责图像的加解密与服务器的交互(例如上传图
  片至数据库或请求检索图片等);
\item \texttt{LogisticPermutation}类负责根据给定的初始密钥求出加解密置
  换表。
\end{itemize}

\subsection{服务端}

\section{功能原理及实现}
\label{sec:func-impl}

为了标准化图片库,便于实现功能,本系统只接受宽$640$像素,高
$480$像素,即$4:3$的灰度JPEG图片。

\subsection{检索匹配方面}
\label{sec:retrieval-impl}
本系统利用二维DCT抽取出一张图片的特征向量。具体方法如下:
\begin{enumerate}
\item 将输入JPEG图片按$8 \times 8$分块,对每块做DCT;
\item 取每块的DC系数,添加到数组$V$中;
\item 将$V$按降序排序,得到$W$,并取$W$的前$n$个元素作为该输入图片的特
  征向量。
\end{enumerate}

其中$n$可以根据实际需要来取。由于$W$中下标越大的元素,其值越小,在向量
距离计算中起到的作用就越小,因此本系统实现中对检索效果和效率进行衡量,
选择了$128$作为该参数的值。

计算距离方面使用的是numpy库中的\texttt{linalg.norm}函数。

\subsection{加解密方面}
\label{sec:enc-dec-impl}

本系统使用的加解密算法的初始密钥由三组$x_0$,$r$,$s$组成,
其中$x_0 \in (-1, 1)$,$r \in [3.57, 4]$,$s > 0, s \in \mathbb{Z}$。

加密算法首先使用伪随机序列生成算法生成若干置乱表,然后根据置乱表对图片
进行块内和块间的置乱达到使图片不可理解、但保留DC系数的统计特性的目的。
解密算法与加密算法类似,只是使用的置乱表不一样,即加解密算法是对合的。

由Logistic映射生成具有$n$个元素的伪随机序列的具体做法\cite{lu2007}是:
\begin{enumerate}
  \item 使用初始密钥$x_0$和$r$,迭代Logistic公式,生成具有$n + s$个元
      素的混沌序列,并舍去前$s$个元素,得到$L = \{x_1, x_2, x_3, \dotsc, x_n\}$;
  \item 对$L$进行升序排序,得到$M = \{x_1^\prime, x_2^\prime, x_3^\prime,
      \dotsc, x_n^\prime\}$;
  \item 定位$x_i$在$x_i^\prime$中的位置序数,得到序数序列,记为$P =
      \{p_1, p_2, p_3, \dotsc, p_n\}$
\end{enumerate}
$P$就是所求的关于初始密钥$x_0$、$r$和$s$的伪随机序列。

由置乱表生成逆置乱表的具体做法是:
\begin{enumerate}
  \item 扩展输入序列为$P^\prime = \{(0, x_1), (1, x_2), (2, x_3),
      \dotsc, (n - 1, x_n)\} & \quad x_i \in P$;
  \item 建立一个新表$I$,使得$\forall (i, x_{i + 1}) \in P^\prime$,
      $I_{x_{i + 1}} = i$
\end{enumerate}
$I$就是所求的关于$P$的逆置乱表。

本系统为了保证每个DCT块的均值不变,利用不同的初始密钥生成了3个伪随机序列,分别是:
\begin{itemize}
  \item 利用第一组初始密钥生成了具有$w$个元素的$T_{column}$,作为块间的列置乱表;
  \item 利用第二组初始密钥生成了具有$h$个元素的$T_{row}$,作为块间的行置乱表;
  \item 利用第三组初始密钥生成了具有64个元素的$T_{block}$,作为DCT块内的置乱表;
\end{itemize}
其中$w$等于图片宽度除以$8$,$h$等于图片宽度除以$8$。

