
\chapter{作品介绍}
\label{chap:intro}

\section{背景分析}
\label{sec:bkg-analysis}

信息技术的高速发展正在越来越快地将我们所居住的现实世界和我们所依赖的信
息世界整合起来。信息技术已经渗透进了我们生活中的点点滴滴:从使用搜索引
擎来检索信息,到在社交网络上分享用户产生内容(User Generated Content),
从个人信息管理(如电子邮件和在线相册),到网络备份服务,云计算时代已
经到来。

基于内容的图像检索系统(Content-based Image Retrieval System)也随着信
息科学技术的进步被发展出来被运用在云计算中,特别地、基于内容的生物特征
图像检索系统(Content-based Biometric Image Retrieval System)对生物特
征图像本身包含的视觉特征和语义信息等进行分析,提取其中能够有效表征图像
的重要特征,并对这些特征进行相似性比较,找出与待检索图像“相似”的生物特
征图像。由于这种基于图像特征的检索方法可以实现图像之间的量化分析与比较、
具有比较强的客观性、能快速有效地进行大量数据的分析与搜索,是一类应用非
常广泛的内容检索系统。

然而信息技术的高速发展也带来了一些不小的威胁。云计算如同一把双刃剑,社
会因云计算的普及而获益匪浅,但个人隐私也无处遁形。其因网络而泄露的事件
时有发生:如2011年3月谷歌邮箱的用户数据泄露事件、2011年4月索
尼PlayStation网络中7700万个注册账户持有人的个人信息失窃和2013年3月云计
算笔记应用Evernote泄露大量用户名、电子邮件地址与加密密码泄露事件等。这
些云计算安全事故给我们敲响了警钟,云安全的保障迫不及待。

在一般的生物特征内容检索系统中,不仅一般信息被存储起来以供后用,应该被
小心保护起来以防范未授权的访问的个人敏感数据(如指纹特征、人脸特征等),
也被按与一般信息同样的方法存储了起来。对于个人敏感信息的安全管理正在成
为一个越来越重要的问题。如果该问题能够得到一定程度的解决,内容检索系统
关于数据保密性和数据可用性的需求将得到一定程度的缓解。

传统的内容检索系统对于敏感信息的保护主要着眼于访问控制(Access Control)
以及安全的数据传输。这两者保证了数据可以安全地送达服务器,而且任何未授
权的人员都不能访问该数据。然而当数据到达服务器后,传统内容检索系统会解
密这些数据,并且直接对明文做归类、检索和数据分析。这就使得敏感信息可能
通过服务端被透露出去,例如敏感信息会对服务端玩忽职守的系统管理员或恶意
的入侵者以明文形式显示出来。

随着隐私信息数量的急剧增加、云计算安全事故的频发和人们保护隐私意识的日
益提高,研发在不牺牲数据可用性和数据可访问性的前提下、能更好地保护个人
敏感信息的检索手段的需求正变得越来越大。在这个趋势下,可以在加密的多媒
体数据库上实现基于内容检索的技术手段,会在帮助人们更快更安全地管理多媒
体数据和隐私中扮演越来越重要的角色。

\section{特色描述}
\label{sec:spec-description}

对服务端内数据的加密能够保护个人隐私信息免疫于服务端不受信任的访问。但
是如果仅使用传统加密算法,服务端将不能顺利处理数据,也就不能对用户的检
索请求产生结果。

基于加密数据库的内容检索,其目标是提供关于加密文档的、高效准确的搜索能
力,并且在搜索时应不需要先解密文档。即服务端仅提供存储和检索功能,不应
有解密个人敏感信息的能力,还应该保证服务端可以从用户数据集所获得的信息
量是最少的。

本系统从JPEG文件的压缩原理入手,实现了一个基于Logistic混沌系统的置乱加
解密算法和基于离散余弦变换的内容检索系统,并基于这两个算法,实现了一
个C/S构架的系统。其中客户端与服务端都由两部分(内核和前端)组成。

\subsection{系统方案特色}
\label{sec:sys-design-spec}
\begin{itemize}
\item 加密内容写回至JPEG文件,使加密内容可见,但不可理解。
\item 加密算法使用的是像素级置换,加解密速度较快。
\item 由于Logistics映射的优良性质\cite{li2011},加密算法的密钥空间至少高于128bit。
\item 禁止指纹相同的图片的多次存储,以防止数据库管理复制图片伪造身份。
\end{itemize}

\subsection{系统实现特色}
\label{sec:sys-impl-spec}
\begin{itemize}
\item 整个系统构建于经过广大用户验证过的开源库或开源程序上,可以保证软
件基础设施的安全性。
\item 系统构架为客户端/服务端,其中交互使用经过广泛使用的HTTP协议,使
一个服务器可以支持多个客户端同时访问,同时不增加服务端的程序复杂性。
\item 服务端使用简洁的Flask框架和支持高并发的tornado库,高效可靠。
\item 客户端使用了跨平台的Qt框架作为前端,内核算法使用了实现了多种快速
算法的OpenCV库和numpy库。
\item 客户端服务端的内核与前端分开,耦合度低,用户可以方便的进行二次开
发。
\end{itemize}

\section{相关工作}
\label{sec:related-work}
为了实现系统设计目标,我们研究了离散余弦变换和Logistic映射的性质及其应用。

\subsection{离散余弦变换}
离散余弦变换(以下简称DCT)是一种将时空域上的函数转换到频域的、具有良好能
量聚集性质的、存在快速算法的数学变换。其公式如下:
\begin{displaymath}
X_k = \frac1 2 \left[x_0 + \left(-1\right)^k x_{N - 1}\right] + \sum_{n = 1}^{N - 2} x_n \cos \left(
        \frac{\pi} {N - 1} n k \right) \qquad k = 0, \dotsc, N - 1
\end{displaymath}

JPEG图片格式利用了人眼对低频信号敏感、对高频信号不敏感的特点对高频信号
进行压缩。DCT使用最广泛的一种变形——二维DCT可以将良好的将高频与低频信
号分离。在JPEG编码中,整个图像数据(灰度值)按$8 \times 8$分块,每块经
过离散余弦变换、量化和熵编码之后按照JPEG标准编码到文件中。

二维DCT(DCT-II)公式如下:
\begin{displaymath}
X_k = \sum_{n = 0}^{N - 1} x_n \cos \left[\frac{\pi} N \left(n +
        \frac1 2\right) k\right] \qquad k = 0, \dotsc, N - 1
\end{displaymath}

在DCT-II中,输入矩阵$I$的能量被充分聚集到了左上角,其中输出矩阵$M$的
$M_{0, 0}$是最能代表输入矩阵的值,这个值被称为DC系数(直流系数),即输入
能量的均值。矩阵中其他的值被称为AC系数(交流系数)。
离DC系数越远的AC系数所代表的频率就越
高,人眼也就对它越不敏感。



\subsection{Logistic映射}
Logistic映射(或称单峰映射)是一种二次的多项式映射,是实际系统中存在的
最简单的非线性差分方程之一\cite{yang2011}。这个映射因生物学家Robert
May在1976年发表的一篇论文而著名。其公式如下:
\begin{displaymath}
x_{n + 1} = rx_n(1 - x_n) \qquad 0 < r \leq 4, r \in \mathbb{R}
\end{displaymath}

Logistic映射所构建的密钥序列具有良好的随机性和初值敏感性,当$r >
3.569945672\dotsb$时,系统进入混沌状态,产生的序列非周期不收敛,随着初
始值的不同有着非常大的差异性\cite{yang2011}。

混沌现象是在非线性动力系统中出现的、具有对初始条件的
敏感依赖性、类噪声、非周期性、确定性的、类随机的过程,这种过程既非周期
又不收敛,其状态完全可以重现\cite{lu2007}。故本系统选用该映射来生成混
沌序列,随后用该混沌序列来生成初始密钥敏感的伪随机序列,
最后利用多个伪随机序列构成置乱矩阵来加密输入图片。

\section{市场分析}
\label{sec:market-analysis}

基于加密数据库的信息检索是一项前景非常好的技术。在当今世界中,摄像头与
数码相机使用得越来越广泛,也就意味着数字多媒体信息的获取已经越来越容易,
而且正在快速渗透入多个方面,如医疗系统、刑侦系统等。

本系统实现了一个可以对固定规格的图像进行加密,并且能对密图进行检索的轻
量级系统。其对生物特征图像的检索效果突出,如人脸图、半身像等。其中加密
算法具有高速、密钥空间较大的特点,检索算法具有快速、简单的特点。可以在
具有大量相同模式的内容需要检索、且内容检索系统需要保证一定秘密性的领域
中使用,如医疗系统中的医疗图像内容检索系统等。

