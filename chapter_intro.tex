
\chapter{作品介绍}
\label{chap:intro}

\section{背景分析}
\label{sec:bkg-analysis}

\section{特色描述}
\label{sec:spec-description}

传统的内容检索系统不能支持对于密文的检索,而传统的加密存储系统不能支持内容检索。
本作品从JPEG文件的压缩算法入手,
实现了一个基于Logistics混沌系统的置乱加解密算法和基于离散余弦变换的内
容检索系统,
并基于这两个算法,实现了一个客户端/服务端构架的系统。
其中客户端/服务端都由两部分(内核和前端)组成。

\subsection{系统方案特色}
\label{sec:sys-design-spec}
\begin{itemize}
\item 加密内容写回至JPEG文件,使加密内容可见,但不可理解。
\item 加密算法使用的是像素置换和,加解密  速度较快。
\item 由于Logistics映射的优良性质\cite{li2011},加密算法的密钥空间至少高于128bit。
\item 向量距离作为内容检索标准,速度较快。
\end{itemize}

\subsection{系统实现特色}
\label{sec:sys-impl-spec}
\begin{itemize}
\item 整个系统构建于经过广大用户验证过的开源库或开源程序上,可以保证软
件基础设施的安全性。
\item 系统构架为客户端/服务端,其中交互使用经过广泛使用的HTTP协议,使
一个服务器可以支持多个客户端同时访问,同时不增加服务端的程序复杂性。
\item 服务端使用简洁的Flask框架和支持高并发的tornado库,高效可靠。
\item 客户端使用了跨平台的Qt框架作为前端,内核算法使用了实现了多种快速
算法的OpenCV库和numpy库。
\item 客户端服务端的内核与前端分开,耦合度低,用户可以方便的进行二次开
发。
\end{itemize}

\section{相关工作}
\label{sec:related-work}

为了实现系统设计目标,我们研究了离散余弦变换和Logistic映射的性质及其应用。

\subsection{离散余弦变换}

离散余弦变换(以下简称DCT)是一种将时空域上的函数转换到频域的、具有良好能
量聚集性质的、存在快速算法的数学变换。其公式如下:
\begin{displaymath}
X_k = \frac1 2 \left[x_0 + \left(-1\right)^k x_{N - 1}\right] + \sum_{n = 1}^{N - 2} x_n \cos \left(
        \frac{\pi} {N - 1} n k \right) & \qquad k = 0, \dotsc, N - 1
\end{displaymath}

JPEG图片格式利用了人眼对低频信号敏感、对高频信号不敏感的特点对高频信号
进行压缩。DCT使用最广泛的一种变形——二维DCT可以将良好的将高频与低频信
号分离。在JPEG编码中,整个图像数据(灰度值)按$8 \times 8$分块,每块经
过离散余弦变换、量化和熵编码之后按照JPEG标准编码到文件中。

二维DCT(DCT-II)公式如下:
\begin{displaymath}
X_k = \sum_{n = 0}^{N - 1} x_n \cos \left[\frac{\pi} N \left(n +
        \frac1 2\right) k\right] & \qquad k = 0, \dotsc, N - 1
\end{displaymath}

在DCT-II中,输入矩阵$I$的能量被充分聚集到了左上角,其中输出矩阵$M$的
$M_{0, 0}$是最能代表输入矩阵的值,这个值被称为DC系数(直流系数),即输入
波形的均值。矩阵中其他的值被称为AC系数(交流系数)。
离DC系数越远的AC系数所代表的频率就越
高,人眼也就对它越不敏感。



\subsection{Logistic映射}

Logistic映射(或称单峰映射)是一种二次的多项式映射,是实际系统中存在的
最简单的非线性差分方程之一\cite{yang2011}。这个映射因生物学家Robert
May在1976年发表的一篇论文而著名。其公式如下:
\begin{displaymath}
x_{n + 1} = rx_n(1 - x_n) & \qquad 0 < r \leq 4, r \in \mathbb{R}
\end{displaymath}

Logistic映射所构建的密钥序列具有良好的随机性和初值敏感性,当$r >
3.569945672\dotsb$时,系统进入混沌状态,产生的序列非周期不收敛,随着初
始值的不同有着非常大的差异性\cite{yang2011}。

混沌现象是在非线性动力系统中出现的、具有对初始条件的
敏感依赖性、类噪声、非周期性、确定性的、类随机的过程,这种过程既非周期
又不收敛,其状态完全可以重现\cite{lu2007}。故本系统选用该映射来生成混
沌序列,随后用该混沌序列来生成初始密钥敏感的伪随机序列,
最后利用多个伪随机序列构成置乱矩阵来加密输入图片。

\section{市场分析}
\label{sec:market-analysis}


